\documentclass[a4paper,12pt]{article}
\usepackage[utf8]{inputenc}
\usepackage{calc}
\usepackage{eso-pic}
\usepackage{graphicx}
\usepackage{parskip}
\usepackage{hyperref}
\usepackage[a4paper,left=20 mm,right=10 mm, bottom=10 mm,top=10 mm,margin = .5in]{geometry} % Adjusting the margins
\usepackage{tikz}
\usepackage{array}
\usepackage{tabularx}
\usepackage{amsmath}
\usepackage{amssymb}
\usetikzlibrary{calc}

\begin{document}

\begin{titlepage}
\begin{tikzpicture}
    [remember picture, overlay]
    \draw[line width = 2pt, black] 
        ($(current page.north west) + (1cm,-1cm)$) 
        rectangle 
        ($(current page.south east) + (-1cm,1cm)$);
\end{tikzpicture}
    \centering
    \vspace*{0 cm}
    \Large\\
    \textbf{Maulana Abul Kalam Azad University of Technology, WB}
    \vspace{0.5cm}
    
    \includegraphics[width=0.35
    \textwidth]{makaut1.jpg} % Assuming you have the logo image
    \vspace{0.5cm}
    
    \LARGE
    \textbf{{Software Tools and Technology\\
        -: Lab Notebook :-}}
    \vspace{0.5cm}
    
    \large
    \textbf{Group 8}
    \vspace{1 cm}
    
    \textbf{Repository Link:} \href{https://github.com/Rajarshi-2005/LaTeX-Group-11}{https://github.com/vardhanbhardvaj/STT-GROUP-8}
    \vspace{1cm}
    
    \textbf{Group Members}
    \vspace{0.5cm}

    \normalsize
    \begin{enumerate}
       \item \textbf{Debashree Bhui - Leader}\\
              \textbf{Roll No: 30001223011}\\
              \textbf{Department: BCA}
         \item \textbf{Harsh Vardhan Pandey}\\
              \textbf{Roll No: 30085323006}\\
              \textbf{Department: BSc in IT (Cyber Security)}
        \item \textbf{Sourish Sarkar}\\
              \textbf{Roll No: 30059223002}\\
              \textbf{Department: BSc in Forensic Science}
        \item \textbf{Jayashri Saha}\\
              \textbf{Roll No: 30059223031}\\
              \textbf{Department: BSc in Forensic Science}
        \item \textbf{Surya sekhar Ghosh}\\
              \textbf{Roll No: 30059223002}\\
              \textbf{Department: BSc in IT (Data Science)}
    \end{enumerate}
    \vspace{0.8 cm}

    \end{titlepage}

    \section*{Table of Contents}
\tableofcontents

\vspace{1cm}

\newpage

\section{Assignment 4: LaTex Mathemaical Notations}
\textbf{By:} Jayshri Saha.


\subsection{Task Overview}

In our recent LaTeX class, we were assigned a task to create a LaTeX document replicating a given attachment and format it according to specific guidelines. The document had to be named as per the following structure:\\ \texttt{Rollno\_DeptName\_Firstname.tex}.\\ 
Additionally, we were required to compile and generate a PDF from the LaTeX source file, include an image file (.png or .jpg), and submit the files in a ZIP archive.

\subsection{Task Instructions}

\begin{itemize}
    \item Write a LaTeX document that replicates the format and structure of a provided attachment.
    \item Name the LaTeX file in the format: \texttt{Rollno\_DeptName\_Firstname.tex}.
    \item Compile the LaTeX file to generate a PDF output.
    \item Include an image in the document, either a .png or .jpg.
    \item Submit the following files:
    \begin{itemize}
        \item LaTeX source code (.tex)
        \item Output document (.pdf)
        \item Image file (.png or .jpg)
    \end{itemize}
    \item Zip the files together into a single archive named as \texttt{Rollno\_DeptName\_Firstname.zip}.
\end{itemize}

\subsection{Writing the LaTeX Source Code}

Using the LaTeX editor, we wrote the document according to the instructions. The task required careful attention to detail to ensure that the layout, text, and structure matched the provided attachment. We also ensured that the formatting adhered to LaTeX best practices, such as using sections, subsections, and appropriate font sizes.

\subsection{Adding an Image}

We included an image file in the LaTeX document using the \texttt{\textbackslash includegraphics} command. The image was placed at the appropriate location in the document, ensuring that it fit well with the text layout and did not disrupt the flow of the document.

\subsection{Compiling to PDF}

After completing the LaTeX code, We compiled the document into a PDF using a LaTeX compiler. This process generated the desired output, replicating the original attachment.



\subsection{Final Submission}

After verifying that all files were correct and the document rendered as intended, we created a ZIP archive with the name \texttt{Rollno\_DeptName\_Firstname.zip}. The ZIP file contained the LaTeX source code, PDF output, and the image file.

\newpage


\section{Assignment 3: Create a CV Using LaTeX}

\textbf{By:} Sourish Sarkar.

\subsection*{Procedure}
\begin{enumerate}
    \item \textbf{Outline Your CV Content:}
    \begin{itemize}
        \item Include your name, contact details, and a professional summary.
        \item List academic qualifications, work experience, skills, projects, and certifications.
    \end{itemize}

    \item \textbf{Decide on the Structure and Layout:}
    \begin{itemize}
        \item Organize the CV into sections such as Personal Information, Experience, Education, etc.
    \end{itemize}

    \item \textbf{Choose a LaTeX Template:}
    \begin{itemize}
        \item Select a template that suits your style from Overleaf or a LaTeX library.
    \end{itemize}

    \item \textbf{Customize the Template:}
    \begin{itemize}
        \item Edit the template with your personal content (experience, qualifications, etc.).
    \end{itemize}

    \item \textbf{Adjust Formatting:}
    \begin{itemize}
        \item Ensure consistency in fonts and section headings.
    \end{itemize}

    \item \textbf{Proofread and Finalize:}
    \begin{itemize}
        \item Review for any errors or formatting issues.
        \item Ensure alignment and organization of sections.
    \end{itemize}

    \item \textbf{Compile and Export:}
    \begin{itemize}
        \item Compile the LaTeX document and export it as a PDF for sharing.
    \end{itemize}
\end{enumerate}


\end{document}