\documentclass[a4paper,12pt]{article}
\usepackage[utf8]{inputenc}
\usepackage{hyperref}
\usepackage{graphicx}
\usepackage[a4paper,left=20 mm,right=10 mm, bottom=10 mm,top=10 mm,margin = .75in]{geometry}\usepackage{tikz}
\usepackage{array}
\usepackage{tabularx}
\usepackage{amsmath}
\usepackage{amssymb}
\usetikzlibrary{calc}

\begin{document}

\begin{titlepage}
\begin{tikzpicture}
    [remember picture, overlay]
    \draw[line width = 2pt, black] 
        ($(current page.north west) + (1cm,-1cm)$) 
        rectangle 
        ($(current page.south east) + (-1cm,1cm)$);
\end{tikzpicture}
    \centering
    \vspace*{0 cm}
    \Large\\
    \textbf{Maulana Abul Kalam Azad University of Technology, WB}
    \vspace{0.5cm}
    
    \includegraphics[width=0.35
    \textwidth]{makaut1.jpg} % Assuming you have the logo image
    \vspace{0.5cm}
    
    \LARGE
    \textbf{{Software Tools and Technology\\
        -: Lab Notebook :-}}
    \vspace{0.5cm}
    
    \large
    \textbf{Group 8}
    \vspace{1 cm}
    
    \textbf{Repository Link: https://github.com/vardhanbhardvaj/STT-GROUP-8}
    \vspace{1cm}
    
    \textbf{Group Members}
    \vspace{0.5cm}

    \normalsize
    \begin{enumerate}
        \item \textbf{Harsh Vardhan Pandey - Leader}\\ 
              \textbf{Roll No: 30085323006}\\   
              \textbf{Department: BSc in IT (Cyber Security)}
        \item \textbf{Debashree Bhui}\\
              \textbf{Roll No: 30001223011}\\
              \textbf{Department: BCA}
        \item \textbf{Sourish Sarkar}\\
              \textbf{Roll No: 30059223002}\\
              \textbf{Department: BSc in Forensic Science}
        \item \textbf{Jayashri Saha}\\
              \textbf{Roll No: 30059223031}\\
              \textbf{Department: BSc in Forensic Science}
        \item \textbf{Surya sekhar Ghosh}\\
              \textbf{Roll No: 30084323020}\\
              \textbf{Department: BSc in IT (Data Science)}
    \end{enumerate}
    \vspace{0.8 cm}

    \end{titlepage}
    \newpage

    \section*{Acknowledgment}
    
    The successful completion of this project would not have been possible without the invaluable support and guidance of our esteemed faculty          members. We extend our sincere gratitude to Dr. Pabitra Pal and Mr. Ayan Ghosh for their mentorship, encouragement, and expertise, which have       been instrumental in our learning and growth.\\

    We would also like to acknowledge the contributions of our group members, Debashree Bhui, Sourish Sarkar, Jayshri Shah, Vardhan Pany and Surya Ghosh. Their dedication, teamwork, and shared knowledge have been invaluable in overcoming challenges and achieving our project goals.\\

Furthermore, we are grateful to the Department of Information Science and Department of Computer Applications for providing us with the necessary resources and facilities to carry out this project.

    \section*{Table of Contents}
\tableofcontents
\newpage

\section{Assignment 1: Calculator Program using C}
\textbf{By:} Debashree Bhui.


\subsection{Objective}
The objective of this lab is to develop a basic calculator program using the C programming language. The calculator will perform simple arithmetic operations like addition, subtraction, multiplication, and division based on user input.

\subsection{Program Overview}
The calculator program is designed to:
\begin{itemize}
    \item Accept two numbers from the user.
    \item Prompt the user to select an arithmetic operation (Addition, Subtraction, Multiplication, Division).
    \item Perform the selected operation.
    \item Display the result of the operation to the user.
\end{itemize}

The program includes error handling to manage division by zero and other invalid inputs.

\subsection{Code Implementation}
The following is the C code for the calculator program:

\begin{verbatim}
#include <stdio.h>

int main() {
    char operator;
    double num1, num2, result;

    printf("Enter an operator (+, -, *, /): ");
    scanf("%c", &operator);

    printf("Enter two operands: ");
    scanf("%lf %lf", &num1, &num2);

    switch(operator) {
        case '+':
            result = num1 + num2;
            break;
        case '-':
            result = num1 - num2;
            break;
        case '*':
            result = num1 * num2;
            break;
        case '/':
            if (num2 != 0)
                result = num1 / num2;
            else {
                printf("Error! Division by zero.\n");
                return -1;
            }
            break;
        default:
            printf("Error! Operator is not correct\n");
            return -1;
    }

    printf("Result: %.2lf\n", result);
    return 0;
}
\end{verbatim}

\subsection{Compiling and Running the Program}
To compile and run the calculator program:
\begin{enumerate}
    \item Open a terminal or command prompt.
    \item Navigate to the directory where the C file is located.
    \item Compile the program using a C compiler (e.g., GCC):
    \begin{verbatim}
    gcc calculator.c -o calculator
    \end{verbatim}
    \item Run the compiled program:
    \begin{verbatim}
    ./calculator
    \end{verbatim}
\end{enumerate}

\subsection{Adding the Calculator Program to GitHub Repository}
To add this calculator program to a GitHub repository, follow these steps:

\subsubsection{Step 1: Initialize a Local Git Repository}
\begin{enumerate}
    \item Open the terminal and navigate to the directory where your \texttt{calculator.c} file is located.
    \item If you haven't already, initialize a Git repository in that directory:
    \begin{verbatim}
    git init
    \end{verbatim}
    This command creates a new Git repository in the current directory.
\end{enumerate}

\subsubsection{Step 2: Add the File to the Repository}
\begin{enumerate}
    \item Add the \texttt{calculator.c} file to the staging area:
    \begin{verbatim}
    git add calculator.c
    \end{verbatim}
    This command stages the file, indicating that you want to include it in the next commit.
\end{enumerate}
\newpage
\subsubsection{Step 3: Commit the Changes}
\begin{enumerate}
    \item Commit the file to the repository with a meaningful message:
    \begin{verbatim}
    git commit -m "Add calculator program in C"
    \end{verbatim}
\end{enumerate}

\subsubsection{Step 4: Push the Changes to GitHub}
\begin{enumerate}
    \item Link your local repository to a remote GitHub repository:
    \begin{verbatim}
    git remote add origin https://github.com/yourusername/your-repo-name.git
    \end{verbatim}
    \item Push the changes to the GitHub repository:
    \begin{verbatim}
    git push -u origin master
    \end{verbatim}
\end{enumerate}

\subsubsection{Step 5: Verify the Upload}
\begin{enumerate}
    \item Go to your GitHub repository URL in a web browser.
    \item Verify that the \texttt{calculator.c} file is listed and accessible in the repository.
\end{enumerate}

\newpage

\section{Assignment 2: Modfying the Submit Button to "CHEEN TAPAK DUM DUM.} 

\textbf{By:} Surya Ghosh.


\subsection{Objective}This document outlines the process of modifying a "Submit" button in a mind reader application and submitting a pull request to the original GitHub repository. The repository in question is available at \url{https://github.com/GeekAyan/STT}. The modification includes renaming the button and fixing proportion issues.


\subsection{Cloning the GitHub Repository}
\textbf{Step}: Clone the GitHub repository using GitHub Desktop.

\textbf{Action}:
\begin{itemize}
    \item Open GitHub Desktop and select \texttt{File > Clone Repository}.
    \item Enter the repository URL: \url{https://github.com/GeekAyan/STT} and select a directory to clone it.
\end{itemize}

\subsection{Opening the Project in an IDE}
\textbf{Step}: Open the cloned project using your preferred IDE (e.g., VS Code, PyCharm).

\textbf{Action}:
\begin{itemize}
    \item Open the folder containing the cloned project.
    \item Review the \texttt{README.md} for instructions on how to run the project.
\end{itemize}

\subsection{Install Dependencies}
\textbf{Step}: Install any dependencies required by the project as per the \texttt{README.md} file.

\textbf{Action}:
\begin{itemize}
    \item Set up the environment. If the project uses Python, create a virtual environment and install dependencies using:
    \begin{lstlisting}[language=bash]
    pip install -r requirements.txt
    \end{lstlisting}
    \item Follow other system requirements mentioned in the \texttt{README.md}.
\end{itemize}

\subsection{Running the Application}
\textbf{Step}: Run the application as per the instructions in \texttt{README.md}.

\textbf{Action}:
\begin{itemize}
    \item Use your IDE's terminal to run the project.
    \item Ensure the application works as expected.
\end{itemize}
\newpage
\subsection{Renaming the Submit Button}
\textbf{Step}: Rename the button from "Submit" to "Chin Tapak Dum Dum."

\textbf{Action}:
\begin{itemize}
    \item Find the code section responsible for the submit button's label.
    \item Modify the label. For example:
    \begin{lstlisting}[language=JAVA]
    <submitButton = new Button("Chin Tapak dum dum");>Chin Tapak Dum Dum</button>
    \end{lstlisting}
\end{itemize}

\subsection{Testing the Application}
\textbf{Step}: Test the application after modifying the button.

\textbf{Action}: 
\begin{itemize}
    \item Run the application again to verify that the button looks correct and functions properly.
\end{itemize}

\subsection{Committing the Changes}
\textbf{Step}: Commit your changes locally.

\textbf{Action}:
\begin{itemize}
    \item Stage the files and commit with a descriptive message, for example:
    \begin{lstlisting}[language=bash]
    git commit -m "Renamed submit button and fixed proportion issue"
    \end{lstlisting}
\end{itemize}

\subsection{Pushing Changes to Your Fork}
\textbf{Step}: Push your changes to your GitHub fork.

\textbf{Action}:
\begin{itemize}
    \item If you haven’t forked the repository, go to the GitHub page and fork it.
    \item Add the forked repository as a remote and push your changes:
    \begin{lstlisting}[language=bash]
    git remote add origin https://github.com/<YourGitHubUsername>/STT.git
    git push origin main
    \end{lstlisting}
\end{itemize}

\subsection{Creating a Pull Request}
\textbf{Step}: Create a pull request to the original repository.

\textbf{Action}:
\begin{itemize}
    \item Go to your fork on GitHub and create a new pull request.
    \item Provide a descriptive title and details of your changes.
    \item Submit the pull request.
\end{itemize}

\newpage
\section{Assignment 3: Create a CV Using LaTeX}

\textbf{By:} Sourish Sarkar.

\subsection*{Procedure}
\begin{enumerate}
    \item \textbf{Outline Your CV Content:}
    \begin{itemize}
        \item Include your name, contact details, and a professional summary.
        \item List academic qualifications, work experience, skills, projects, and certifications.
    \end{itemize}

    \item \textbf{Decide on the Structure and Layout:}
    \begin{itemize}
        \item Organize the CV into sections such as Personal Information, Experience, Education, etc.
    \end{itemize}

    \item \textbf{Choose a LaTeX Template:}
    \begin{itemize}
        \item Select a template that suits your style from Overleaf or a LaTeX library.
    \end{itemize}

    \item \textbf{Customize the Template:}
    \begin{itemize}
        \item Edit the template with your personal content (experience, qualifications, etc.).
    \end{itemize}

    \item \textbf{Adjust Formatting:}
    \begin{itemize}
        \item Ensure consistency in fonts and section headings.
    \end{itemize}

    \item \textbf{Proofread and Finalize:}
    \begin{itemize}
        \item Review for any errors or formatting issues.
        \item Ensure alignment and organization of sections.
    \end{itemize}

    \item \textbf{Compile and Export:}
    \begin{itemize}
        \item Compile the LaTeX document and export it as a PDF for sharing.
    \end{itemize}
\end{enumerate}

\newpage

\section{Assignment 4: LaTex Mathemaical Notations}
\textbf{By:} Jayshri Saha.


\subsection{Task Overview}

In our recent LaTeX class, we were assigned a task to create a LaTeX document replicating a given attachment and format it according to specific guidelines. The document had to be named as per the following structure:\\ \texttt{Rollno\_DeptName\_Firstname.tex}.\\ 
Additionally, we were required to compile and generate a PDF from the LaTeX source file, include an image file (.png or .jpg), and submit the files in a ZIP archive.

\subsection{Task Instructions}

\begin{itemize}
    \item Write a LaTeX document that replicates the format and structure of a provided attachment.
    \item Name the LaTeX file in the format: \texttt{Rollno\_DeptName\_Firstname.tex}.
    \item Compile the LaTeX file to generate a PDF output.
    \item Include an image in the document, either a .png or .jpg.
    \item Submit the following files:
    \begin{itemize}
        \item LaTeX source code (.tex)
        \item Output document (.pdf)
        \item Image file (.png or .jpg)
    \end{itemize}
    \item Zip the files together into a single archive named as \texttt{Rollno\_DeptName\_Firstname.zip}.
\end{itemize}

\subsection{Writing the LaTeX Source Code}

Using the LaTeX editor, we wrote the document according to the instructions. The task required careful attention to detail to ensure that the layout, text, and structure matched the provided attachment. We also ensured that the formatting adhered to LaTeX best practices, such as using sections, subsections, and appropriate font sizes.

\subsection{Adding an Image}

We included an image file in the LaTeX document using the \texttt{\textbackslash includegraphics} command. The image was placed at the appropriate location in the document, ensuring that it fit well with the text layout and did not disrupt the flow of the document.

\subsection{Compiling to PDF and Submission}

After completing the LaTeX code, We compiled the document into a PDF using a LaTeX compiler. This process generated the desired output, replicating the original attachment.After this we need to sumbit it in zip file conatining all images and tex file


\newpage

\section{Git Branching Merging and Conflict Resolution}
\textbf{By:} Harsh Vardhan Pandey.

\subsection{Objective}
The objective of this assignment was to demonstrate proficiency in Git branching, merging, and conflict resolution by working through a series of steps involving branch creation, modification of shared files, and the resolution of conflicts that arise from merging.


\subsection{ Creating a New Repository}
We started by creating a new repository on GitHub named \texttt{git-advanced}. We cloned this repository to our local machine.

\subsection{ Creating and Working on Branches}
Next, we created a new branch called \texttt{feature-1} and switched to it. After switching to \texttt{feature-1}, we created a file named \texttt{shared.txt} and added the following content:
\begin{verbatim}
This is a shared file.
Line 1: Original text.
Line 2: Original text.
\end{verbatim}
We staged and committed the changes and then pushed the branch to GitHub.

\subsection{Creating a Second Branch from feature-1}
Following this, we created another branch called \texttt{feature-2} and switched to it. In this branch, we modified the second line of \texttt{shared.txt} to.Modifying the second line of feature-2:
\begin{verbatim}
Line 2: Modified text in feature-2.
\end{verbatim}
We staged and committed the changes and then pushed the \texttt{feature-2} branch to GitHub.

\subsection{Modifying the File in Feature-1 Branch}
We switched back to \texttt{feature-1} and then modified the second line of \texttt{shared.txt} to:
\begin{verbatim}
Line 2: Modified text in feature-1.
\end{verbatim}
We staged and committed the changes and pushed this branch to GitHub.

\begin{center}
\includegraphics[width=0.7
    \textwidth]{image1.png}
    \vspace{0.5cm}
\end{center}

\subsection{Merging Feature-1 into Main}
After ensuring the changes in \texttt{feature-1} were pushed, we switched to the \texttt{main} branch and merged \texttt{feature-1} into it and pushed the updated \texttt{main} branch.

\begin{center}
\includegraphics[width=0.7
    \textwidth]{image2.png} 
    \vspace{0.5cm}
\end{center}
\subsection{Merging Feature-2 and Resolving Conflict}
Next, we attempted to merge \texttt{feature-2} into the \texttt{main} branch. This resulted in a merge conflict in \texttt{shared.txt} because both branches modified the second line. Git identified the conflict and marked it as follows:
\begin{verbatim}
<<<<<<< HEAD
Line 2: Modified text in feature-1.
=======
Line 2: Modified text in feature-2.
>>>>>>> feature-2
\end{verbatim}
We resolved the conflict by editing the file to the following content:
\begin{verbatim}
Line 2: Merged text from both branches.
\end{verbatim}
We then staged and committed the conflict resolution. Finally, we pushed the changes to GitHub.\\

\begin{center}
     \includegraphics[width=0.7
    \textwidth]{image4.png} 
    \vspace{0.5cm}
\end{center}

\subsection*{7. Deleting Branches}
After the merge and conflict resolution, we deleted both feature branches locally and remotely.
\begin{center}
 \includegraphics[width=0.4
    \textwidth]{image6.png}
    \vspace{0.5cm}
     \includegraphics[width=0.4
    \textwidth]{image7.png}
    \vspace{0.5cm}
\end{center}
\begin{center}
\includegraphics[width=0.7
    \textwidth]{image9.png} 
    \vspace{0.5cm}
\end{center}
\newpage
\section{Conclusion}

This project has been a valuable learning experience, allowing us to apply our knowledge and skills in a practical setting. We have successfully achieved the objectives outlined at the beginning of this project, and we believe that the outcomes will be beneficial in our future endeavors.\\

Through this project, we have gained a deeper understanding of Git, GitHub and Latex. We have also developed important skills such as Version Control System, Industry level collaboration which will be invaluable in our academic and professional careers.
\end{document}
